\documentclass[a4paper,12pt]{article}
\usepackage[utf8]{inputenc}
\usepackage[spanish]{babel}
\usepackage{listings}
\usepackage{xcolor}

%opening
\title{Tarea No. 14. ¿Qué es I18N y L10N?}
\author{Barrera Pérez Carlos Tonatihu \\ Profesor: José Asunción Enríquez 
Zárate \\ Web Application Development \\ Grupo: 3CM9 }

\begin{document}

\maketitle
\newpage
\section{Introducción}
Los términos i18n y l10n generan bastante confusión y suelen ser mal usados. 
Estos dos están relacionados pero no tienen el mismo significado, estos se 
refieren a diferentes niveles de implementación y de experiencia por lo que es 
necesario conocerlos para saber en que momento utilizar cada termino.
\section{Desarrollo}
i18n es el proceso de asegurarse que el software es lo suficientemente flexible 
y listo para soportar múltiples posibles zonas, en un inicio ningún lenguaje 
esta involucrado en este proceso, una aplicación podría nunca ser traducida 
pero si es adecuadamente internacionalizada entonces su traducción sera 
bastante sencilla, por lo que la internacionalización es hacer que el software 
sea amigable con la zona.

Localización por el otro lado es traducir a un lenguaje en especifico. En un 
inicio la localización no requiere internacionalización para ser hecha, sin 
embargo, si la aplicación no ha sido internacionalizada, entonces eso significa 
que el proceso de localización involucrara bastantes modificaciones a bajo 
nivel en el código.

Asi que l10n sin i18n es malo, sin embargo en la actualidad esto es algo que ya 
no es tan común.

Hay varios aspectos involucrados en i18n, algunos son más comunes que otros:

\begin{itemize}
 \item Capacidad para desplegar o imprimir texto.
 \begin{itemize}
  \item Soportar diferentes fuentes.
 \end{itemize}
 \item Capacidad para manipular texto.
 \begin{itemize}
  \item Capacidad para manipular cadenas.
  \item Soportar capacidades como copiar, pegar y cortar.
  \item Búsqueda de texto.
 \end{itemize}
 \item Capacidad para insertar texto.
 \item Capacidad para manipular o presentar fechas.
 \item Capacidad para definir o manipular configuraciones por zona.
 \begin{itemize}
  \item Diferentes unidades, monedas, etc.
  \item Diferentes formas de desplegar números.
 \end{itemize}
\end{itemize}

Por otro lado, en la localización los siguientes aspectos son importantes:

\begin{itemize}
 \item Traducir datos de texto traducible.
 \item Proporcionar datos de texto localizados únicos.
 \item Correctores ortográficos.
 \item Distribuciones de teclado.
 \item Métodos de entrada.
 \item Reglas de clasificación.
\end{itemize}

\section{Conclusiones}
Como se mostró en este trabajo, el conocer en que consiste i18n y l10n, así 
como sus diferencias permite resolver diversos problemas de acuerdo a sus 
necesidades.

Finalmente, se puede concluir que l10n se aplica en problemas de traducción de 
un lenguaje a otro, así que los traductores son los que más se involucran en 
este caso; por otro lado, i18n es utilizado en problemas de infraestructura; 
por lo que afecta varios lenguajes, en este caso los desarrolladores son los 
principales involucrados.
\end{document}
