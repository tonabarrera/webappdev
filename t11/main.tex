\documentclass[a4paper,12pt]{article}
\usepackage[utf8]{inputenc}
\usepackage[spanish]{babel}
\usepackage{listings}
\usepackage{xcolor}
\definecolor{gris}{RGB}{123, 126, 132}
\definecolor{morado}{RGB}{81, 40, 155}
\definecolor{amarillo}{RGB}{253,151,31}
\definecolor{magenta}{RGB}{249,38,114}

\lstdefinestyle{customJava}{
    frame=tb,
    language=Java,
    backgroundcolor=\color{white},   
    commentstyle=\itshape\color{gris},
    keywordstyle=\bfseries\color{magenta},
    numberstyle=\color{morado},
    stringstyle=\color{amarillo},
    identifierstyle=\color{black},
    basicstyle=\footnotesize,
    breakatwhitespace=false,         
    breaklines=true,                 
    captionpos=b,
    keepspaces=true,                 
    numbers=left,                    
    numbersep=5pt,                  
    showspaces=false,                
    showstringspaces=false,
    showtabs=false,                  
    tabsize=2,
}

\renewcommand{\lstlistingname}{Código}

\title{Tarea No. 11. Singleton}
\author{Barrera Pérez Carlos Tonatihu \\ Profesor: José Asunción Enríquez 
Zárate \\ Web Application Development \\ Grupo: 3CM9 }

\begin{document}

\maketitle
\newpage
\tableofcontents
\newpage

\section{Introducción}
El patrón singleton es un patrón de diseño que restringe las instancias de una 
clase a un objeto. Existen diferentes formas de implementarlo, para esto es 
necesario tener conocimiento sobre variables de clase estáticas y modificadores 
de acceso.

\section{Desarrollo}
En programación orientada a objetos una clase singleton es una clase que tiene 
solo un objeto a la vez. Después de la primera inicialización si se intenta 
inicializar de nuevo, la nueva variable también apuntara a la primera instancia 
creada. Así que cualquier modificación que se haga a cualquier variable dentro 
de la clase a través de la instancia, afectara la variable de la instancia 
única.

Para diseñar una clase singleton es necesario:
\begin{itemize}
 \item Hacer el constructor privado.
 \item Escribir un método estático que retorne el tipo de objeto de la clase 
singleton. Aquí se aplica el concepto de inicialización perezosa.
\end{itemize}

\subsection{Implementación clásica}
Esta es la implementación más conocida, solo se tiene una instancia estatica 
de la clase cuando se crea, sin embargo, presenta el grave problema de que no 
es seguro en hilos. Ya que de existir dos o más hilos se podrian crear dos 
objetos del singleton.

\begin{lstlisting}[language=Java,style=customJava,basicstyle=\fontfamily{cmss}
\small]
class Singleton { 
    private static Singleton obj; 
 
    /*
    * Constructor privado para forzar a utilizar el metodo
    * getInstance() para crear el objeto Singleton
    */
    private Singleton() {} 
  
    public static Singleton getInstance() { 
        if (obj==null) 
            obj = new Singleton(); 
        return obj; 
    } 
}
\end{lstlisting}

\subsection{Implementación de hilo sincronizado}
Esta implementación es segura ya que se utiliza el modificador synchronized, 
por lo que solo un hilo puede ejecutar el método a la vez. La desventaja es que 
esta implementación es bastante costosa en cuanto al desempeño; pero si este no 
es un problema es una buena solución.

\begin{lstlisting}[language=Java,style=customJava,basicstyle=\fontfamily{cmss}
\small]
class Singleton { 
    private static Singleton obj; 
  
    private Singleton() {} 
  
    // Solo un hilo puede ejecutar este metodo a la vez
    public static synchronized Singleton getInstance() { 
        if (obj==null) 
            obj = new Singleton(); 
        return obj; 
    } 
}
\end{lstlisting}

\subsection{Inicialización impaciente}
En esta opción, se inicializa la variable al cargar la clase, la JVM ejecuta el 
inicializador estático y por lo tanto es seguro en hilos. Se usa este método 
cuando la clase singleton es ligera y es usado a lo largo de la ejecución del 
programa.

\begin{lstlisting}[language=Java,style=customJava,basicstyle=\fontfamily{cmss}
\small]
class Singleton { 
    private static Singleton obj = new Singleton(); 
  
    private Singleton() {} 
  
    public static Singleton getInstance() { 
        return obj; 
    } 
} 

\end{lstlisting}

\subsection{Double checked locking}
Una vez qu un objeto es creado, la sincronización deja de ser útil debido a que 
ahora el objeto no sera nulo y cualquier secuencia de operaciones conducirá a 
resultados consistentes.

Por lo que solo tendremos bloqueo en el método getInstance() una sola vez, 
cuando el objeto es nulo. En esta forma solo se sincroniza una sola vez.

Por ultimo, el objeto se declara volatil, lo cual asegura que múltiples hilos 
brinden la variable objeto correctamente cuando es inicializado.
\begin{lstlisting}[language=Java,style=customJava,basicstyle=\fontfamily{cmss}
\small]
class Singleton { 
    private volatile static Singleton obj; 
  
    private Singleton() {} 
  
    public static Singleton getInstance() { 
        if (obj == null) { 
            // Para hacerlo el hilo seguro
            synchronized (Singleton.class) { 
                /*
                * Se vuelve a checar ya que varios hilos
                * pueden alcanzar el paso anterior
                */
                if (obj==null) 
                    obj = new Singleton(); 
            } 
        } 
        return obj; 
    } 
} 

\end{lstlisting}

\section{Conclusiones}
Es conocer los requerimientos de la aplicación que se este desarrollando es 
importante, ya que como se muestra en esta tarea existen diversas formas de 
implementar este patrón, así resulta bastante útil el saber cual es el que más 
se ajusta a nuestras necesidades.

\end{document}
