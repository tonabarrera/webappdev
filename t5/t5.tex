\documentclass[12pt, titlepage]{article}
\usepackage[letterpaper, margin=3cm]{geometry}
\usepackage[utf8]{inputenc}
\usepackage[spanish]{babel}
\usepackage{graphicx}
\usepackage{url}
\usepackage{float}
\usepackage{listings}

\usepackage{xcolor}
\definecolor{gris}{RGB}{123, 126, 132}
\definecolor{morado}{RGB}{157, 101, 255}
\definecolor{amarillo}{RGB}{253,151,31}
\definecolor{magenta}{RGB}{249,38,114}

\lstdefinestyle{customXML}{
    frame=tb,
    language=XML,
    basicstyle=\footnotesize\ttfamily,
    backgroundcolor=\color{white},   
    commentstyle=\itshape\color{gris},
    keywordstyle=\bfseries\color{magenta},
    numberstyle=\color{morado},
    stringstyle=\color{amarillo},
    identifierstyle=\color{magenta},
    basicstyle=\footnotesize,
    breakatwhitespace=false,         
    breaklines=true,                 
    captionpos=b,
    keepspaces=true,                 
    numbers=left,                    
    numbersep=5pt,                  
    showspaces=false,                
    showstringspaces=false,
    showtabs=false,                  
    tabsize=2,
    morekeywords={web-app, encoding, xmlns, xmlns:xsi, xsi:schemaLocation}
}

%opening
\title{Tarea No. 5. Archivo web.xml}
\author{Barrera Pérez Carlos Tonatihu \\ Profesor: José Asunción Enríquez Zárate 
\\ Web Application Development \\ Grupo: 3CM9 }

\begin{document}

\maketitle
\newpage
\tableofcontents
\newpage

\section{Introducción}
Las aplicaciones web desarrolladas con java usan un descriptor de despliegue 
para determinar como serán mapeadas las URLs de los servlets, que URLs 
requerirán autenticación y otra información.

El archivo se llama web.xml y se ubica dentro de la carpeta 
WEB-INF/ dentro del proyecto \cite{google}.

\section{Desarrollo}
La estructura mínima que contiene este archivo es el descriptor de apertura y 
cierre $<$web-app$>$, el siguiente es un ejemplo que funciona en la versión 6 
de Tomcat.

\begin{lstlisting}[language=XML, style=customXML]
<?xml version="1.0" encoding="ISO-8859-1"?>
<web-app xmlns="http://java.sun.com/xml/ns/j2ee"
  xmlns:xsi="http://www.w3.org/2001/XMLSchema-instance"
  xsi:schemaLocation="http://java.sun.com/xml/ns/javaee
  http://java.sun.com/xml/ns/javaee/web-app_2_5.xsd"
  version="2.5">
</web-app>
\end{lstlisting}

Los descriptores que puede contener este archivo son los siguientes:

\begin{lstlisting}[language=XML, style=customXML]
<?xml version="1.0" encoding="UTF-8"?>
<web-app xmlns="http://java.sun.com/xml/ns/j2ee"
  xmlns:xsi="http://www.w3.org/2001/XMLSchema-instance"
  xsi:schemaLocation="http://java.sun.com/xml/ns/j2ee 
  http://java.sun.com/xml/ns/j2ee/web-app_2_4.xsd"
  version="2.4">

  <!-- Nombre de la aplicacion -->
  <display-name>Nombre</display-name>
  <description>Descripcion</description>
  
  <!-- ========================================================== -->
  <!-- Servlets -->
  <!-- ========================================================== -->

  <servlet>
    <servlet-name>cms</servlet-name>
    <servlet-class>com.metawerx.servlets.ContentManagementSystem</servlet-class>
    <description>Descripcion</description>

    <!--Parametros del servlet-->
    <init-param>
      <param-name>debug</param-name>
      <param-value>true</param-value>
    </init-param>

    <!--Carga este servlet cuando la aplicacion empieza-->
    <load-on-startup>5</load-on-startup>
  </servlet>

  <!-- Mapea las urls -->
  <servlet-mapping>
  <!-- Para cualquier url que termine en *.cms-->
    <servlet-name>cms</servlet-name>
    <url-pattern>*.cms</url-pattern>
  </servlet-mapping>

  <!-- Define un JSP que responda a las urls /shop/item/* -->
  <servlet>
    <servlet-name>pathjsp</servlet-name>
    <jsp-file>pathfinder.jsp</jsp-file>
  </servlet>

  <servlet-mapping>
    <servlet-name>pathjsp</servlet-name>
    <url-pattern>/shop/item/*</url-pattern>
  </servlet-mapping>

  <!-- ========================================================== -->
  <!-- Seguridad -->
  <!-- ========================================================== -->

  <!-- Roles  -->
  <security-role>
    <role-name>admin</role-name>
  </security-role>
  
  <!-- Restriccion para /private/* -->
  <security-constraint>
    <display-name>Nombre que se muestra</display-name>
    <web-resource-collection>
      <web-resource-name>Protected Area</web-resource-name>
      <url-pattern>/private/*</url-pattern>
                        
      <!-- Metodos que se protegen -->
      <http-method>DELETE</http-method>

    </web-resource-collection>

    <auth-constraint>
    <!-- Quienes pueden acceder -->
      <role-name>admin</role-name>
    </auth-constraint>

  </security-constraint>

  <!-- ========================================================== -->
  <!-- Manejador de errrores -->
  <!-- ========================================================== -->
  <error-page>
    <error-code>404</error-code>
    <location>/error404.jsp</location>
  </error-page>
  
  <error-page>
    <exception-type>java.lang.Throwable</exception-type>
    <location>/errorThrowable.jsp</location>
  </error-page>

</web-app> 
\end{lstlisting}


\section{Conclusiones}
El conocer y entender el funcionamiento del archivo web.xml es fundamental en 
el desarrollo de aplicaciones web con Java EE, he incluso en frameworks que se 
basen en Java para el desarrollo debido a que es la base del como funciona una 
aplicación de este tipo. Tal vez no se utilicen todos los descriptores que se 
pueden declarar en este archivo, eso depende de lo que se requiera; pero el 
simple hecho de saber que existen permite que el desarrollo pueda ser más ágil.

\bibliographystyle{ieeetr}
\bibliography{referencias}

\end{document}
