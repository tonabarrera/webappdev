\documentclass[a4paper,12pt]{article}
\usepackage[utf8]{inputenc}
\usepackage[spanish]{babel}
\usepackage{graphicx}
\usepackage{float}


%opening
\title{Tarea No. 2. Diferencias entre servidor web, contenedor web y servidor de aplicaciones}
\author{Barrera Pérez Carlos Tonatihu \\ Profesor: José Asunción Enríquez Zárate \\ Web Application Development \\ Grupo: 3CM9 }

\begin{document}

\maketitle
\newpage
\tableofcontents

\newpage
\section{Introducción}
Este trabajo recopila información sobre lo que es un servidor web, contenedor 
web y un servidor de aplicaciones y enlista las principales diferencias que 
existen este estas tres tecnologías.

\section{Desarrollo}
Para poder entender las diferencias primero se debe de entender que es un servidor web, un contenedor web y un servidor de aplicaciones.
\subsection{Servidor web}
Un servidor web o servidor HTTP es un programa informático que procesa 
una aplicación del lado del servidor, realizando conexiones bidireccionales o 
unidireccionales y síncronas o asíncronas con el cliente y generando o cediendo 
una respuesta en cualquier lenguaje o Aplicación del lado del cliente. El código 
recibido por el cliente es renderizado por un navegador web. Para la transmisión 
de todos estos datos suele utilizarse algún protocolo. Generalmente se usa el 
protocolo HTTP para estas comunicaciones, perteneciente a la capa de aplicación 
del modelo OSI \cite{wiki}.
Algunos ejemplos de este tipo de herramienta son:
\begin{enumerate}
 \item Apache HTTP Server
 \item Microsoft IIS
 \item NGINX
\end{enumerate}

\subsection{Contenedor web}
Un contenedor Web (más comúnmente conocido como un contenedor de servlets) es 
una aplicación implementada en los servidores web que hace que la implementación 
de “Java Servlets” y posible de Java Server Pages “. Un contenedor crea un 
ambiente totalmente independiente para el funcionamiento de los servlets y las 
páginas de servidor Java con el fin de ofrecer contenido dinámico a los 
visitantes del sitio. Está diseñado principalmente para ejecutar aplicaciones 
Java de codificación en un servidor web. Todos los contenedores web son JEE 
(Java Platform Enterprise Edition) compatible. Los servlets se ejecutan en el 
entorno de tiempo de ejecución proporcionado por el contenedor web mediante el 
uso de motor de JSP y motores de servlets. \cite{jee}.
\begin{enumerate}
 \item Apache Tomcat
\end{enumerate}

\subsection{Servidor de aplicaciones}
En informática, se denomina servidor de aplicaciones a un servidor en una red de 
computadores que ejecuta ciertas aplicaciones. Usualmente se trata de 
un dispositivo de software que proporciona servicios de aplicación a las 
computadoras cliente. Un servidor de aplicaciones generalmente gestiona la mayor 
parte (o la totalidad) de las funciones de lógica de negocio y de acceso a los 
datos de la aplicación. Los principales beneficios de la aplicación de la 
tecnología de servidores de aplicación son la centralización y la disminución de 
la complejidad en el desarrollo de aplicaciones.
Algunos ejemplos de este tipo de herramienta son:
\begin{enumerate}
 \item Oracle WebLogic
 \item WebSphere
 \item EAServer
 \item JBoss
 \item TomEE
 \item GlassFish
\end{enumerate}

\subsection{Diferencias}
Al tener las definiciones de estas tres herramientas se pueden concluir
las siguientes diferencias.

\begin{enumerate}
 \item Un servidor web está diseñado para servir contenido HTTP. Servidor de 
aplicaciones puede servir contenido HTTP pero no se limita a sólo HTTP. Se puede 
proporcionar otro soporte de protocolo como RPC/RMI.
 \item Un servidor web en su mayoría está diseñado para servir contenido estático.
 \item Un servidor de aplicaciones tiende a referirse a un marco de nivel superior utilizado para generar contenido dinámico.
 \item Un servidor de aplicaciones agrega más operaciones que el servidor web, como el administrador de transacciones, la fuente de datos y otras operaciones comerciales complicadas.
\end{enumerate}


\section{Conclusiones}
El conocer las diferencias entre servidor web, contenedor web y servidor de 
aplicaciones es fundamental para elegir la herramienta que más se ajuste a 
nuestras necesidades y no utilizar algo que consuma demasiados recursos en un 
proyecto pequeño o viceversa. Además de que el conocer estas herramientas nos 
permite evitar confusiones que se puedan generar al plantear el desarrollo de 
algún proyecto.

\bibliographystyle{ieeetr}
\bibliography{referencias}

\end{document}
