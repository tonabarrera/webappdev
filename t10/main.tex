\documentclass[a4paper,12pt]{article}
\usepackage[utf8]{inputenc}
\usepackage[spanish]{babel}
\usepackage{url}

%opening
\title{Tarea No. 10. ORM}
\author{Barrera Pérez Carlos Tonatihu \\ Profesor: José Asunción Enríquez 
Zárate \\ Web Application Development \\ Grupo: 3CM9 }

\begin{document}

\maketitle
\newpage
\tableofcontents
\newpage

\section{Introducción}
La presente tarea es una investigación sobre el Mapeo Objeto-Relacional (ORM 
por sus siglas en inglés), sus ventajas y desventajas y con información general 
sobre este tema.
\section{Desarrollo}
ORM es un mecanismo que hace posible acceder y manipular objetos sin tener que 
considerar como esos objetos están relacionados a la fuente de datos. Los ORM 
permiten a los programadores mantener la consistencia en los objetos a través 
del tiempo incluso cuando las fuentes de las que provienen, que las consumen o 
que las aplicaciones que acceden a ellos cambia. \cite{search}

\subsection{Ventajas y desventajas de los ORM}
Ventajas
\begin{itemize}
 \item No tienes que escribir código SQL en la mayoría de los casos.
 \item Te dejan sacar partido a las bondades de la programación orientada a 
objetos.
 \item Nos permiten aumentar la reutilización del código y mejorar el 
mantenimiento del mismo, ya que tenemos un único modelo en un único lugar, 
que podemos reutilizar en toda la aplicación, y sin mezclar consultas con 
código o mantener sincronizados los cambios de la base de datos con el modelo 
y viceversa. 
 \item Hacen muchas cosas por nosotros: desde el acceso a los datos, hasta 
la conversión de tipos o la internacionalización del modelo de datos. \cite{orm}
\end{itemize}
Desventajas
\begin{itemize}
 \item Pueden llegar a ser muy complejos.
 \item No son ligeros por regla general.
 \item Si no sabes bien lo que estás haciendo y las implicaciones que tiene en 
el modelo relacional puedes construir modelos que generen consultas monstruosas 
y muy poco óptimas contra la base de datos, agravando el problema del 
rendimiento y la eficiencia.
 \item La configuración inicial que requieren se puede complicar dependiendo 
de la cantidad de entidades que se manejen y su complejidad, del gestor de 
datos subyacente, etc. \cite{orm}
\end{itemize}

Al trabajar con Java existen varias opciones a considerar, el ORM más popular 
es Hibernate de Red Hat. Hay muchos otros como Jooq, ActiveJDBC que trata de 
emular los Active Records de Ruby On Rails, o QueryDSL, pero en realidad ninguno 
llega ni por asomo al nivel de uso de Hibernate.

\section{Conclusiones}
El utilizar ORM puede llegar a ser un arma de doble filo, debido a que en un 
inicio parece facilitar el desarrollo de aplicación al hacer el acceso de datos 
una parte más homogénea respecto al resto de desarrollo que se tiene; pero si 
no se tiene la suficiente experiencia con estas herramientas pude resultar en 
grandes problemas en la aplicación.

Es por esto que la elección de usar o no debe de estar bien fundamentada y no 
simplemente usar un ORM porque se ve bonito.

\bibliographystyle{ieeetr}
\bibliography{referencias}
\end{document}
