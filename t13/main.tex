\documentclass[a4paper,12pt]{article}
\usepackage[utf8]{inputenc}
\usepackage[spanish]{babel}
\usepackage{listings}
\usepackage{xcolor}
\definecolor{gris}{RGB}{123, 126, 132}
\definecolor{morado}{RGB}{81, 40, 155}
\definecolor{amarillo}{RGB}{253,151,31}
\definecolor{magenta}{RGB}{249,38,114}

\lstdefinestyle{customXML}{
    frame=tb,
    language=XML,
    backgroundcolor=\color{white},   
    commentstyle=\itshape\color{gris},
    keywordstyle=\bfseries\color{magenta},
    numberstyle=\color{morado},
    stringstyle=\color{amarillo},
    identifierstyle=\color{black},
    basicstyle=\footnotesize,
    breakatwhitespace=false,         
    breaklines=true,                 
    captionpos=b,
    keepspaces=true,                 
    numbers=left,                    
    numbersep=5pt,                  
    showspaces=false,                
    showstringspaces=false,
    showtabs=false,                  
    tabsize=2,
}

\renewcommand{\lstlistingname}{Código}
%opening
\title{Tarea No. 13. Archivo persistence.xml}
\author{Barrera Pérez Carlos Tonatihu \\ Profesor: José Asunción Enríquez 
Zárate \\ Web Application Development \\ Grupo: 3CM9 }

\begin{document}

\maketitle
\newpage
\section{Introducción}
El archivo persistence.xml es la parte fundamental en la configuración de JPA. 
Este archivo define una o mas unidades de persistencia en las cuales puedes 
configurar parámetros como:
\begin{itemize}
 \item El nombre de la unidad de persistencia.
 \item Cuales seran las clases de persistencia que son parte de la unidad de 
persistencia.
\item El como las clases son mapeadas en la base de datos.
\item El proveedor de persistencia que sera utilizado.
\item La fuente de datos que se utilizara para conectarse a la base de datos.
\item Como crear y validar el esquema de la base de datos.
\item El modo del cache de segundo nivel.
\item Muchas configuraciones especificas de proveedor.
\end{itemize}

Es por esto que en esta tarea se exploraran los principales parámetros de 
configuración que se tienen en este archivo.

\section{Desarrollo}
El siguiente es un ejemplo de un archivo persistence.xml con la respectiva 
descripción de cada atributo y etiqueta.

\begin{lstlisting}[language=XML,style=customXML]
<persistence xmlns="http://java.sun.com/xml/ns/persistence"
    xmlns:xsi="http://www.w3.org/2001/XMLSchema-instance"
    xsi:schemaLocation="http://java.sun.com/xml/ns/persistence 
http://java.sun.com/xml/ns/persistence/persistence_2_0.xsd"
    version="2.0">
   <persistence-unit name="manager1" transaction-type="JTA">
      <provider>org.hibernate.ejb.HibernatePersistence</provider>
      <jta-data-source>java:/DefaultDS</jta-data-source>
      <mapping-file>ormap.xml</mapping-file>
      <jar-file>MyApp.jar</jar-file>
      <class>org.acme.Employee</class>
      <class>org.acme.Person</class>
      <class>org.acme.Address</class>
      <shared-cache-mode>ENABLE_SELECTOVE</shared-cache-mode>
      <validation-mode>CALLBACK</validation-mode>
      <properties>
         <property name="hibernate.dialect" 
value="org.hibernate.dialect.HSQLDialect"/>
         <property name="hibernate.hbm2ddl.auto" value="create-drop"/>
      </properties>
   </persistence-unit>
</persistence>

\end{lstlisting}

\begin{itemize}
 \item \textbf{name}. Este atributo defiene el nombre del entity manager.
 \item \textbf{transaction-type}. Este atributo define el tipo de transacción a 
utilizar. Ya sea JTA o RESOURCE\_LOCAL, JTA es la que se utiliza por defecto en 
un ambiente JavaEE y RESOURCE\_LOCAL en JavaSE.
\item \textbf{provider}. Es el nombre completo del proveedor de persistencia 
EJB.
\item \textbf{jta-data-source, non-jta-data-source}. Este es el nombre del JDNI 
donde se ubica el DataSource.
\item \textbf{mapping-file}. El elemento de clase que especifica un archivo de 
asignación XML compatible con EJB3 que se asignara.
\item \textbf{jar-file}. Especifica un jar a analizar. Todos los elementos de 
este jar son agregados a la configuración de la unidad de persistencia. Este 
elemento es principalmente utilizado in JavaEE.
\item \textbf{exclude-unlisted-classes}. No revisa el archivo jar principal 
para las clases anotadas. Solo las clases explicitas serán parte de la unidad 
de persistencia.
\item \textbf{class}. Especifica el nombre completo de la clase que se va a 
mapear.
\item \textbf{shared-cache-mode}. Por defecto, las entidades son elegidas para 
el segundo nivel de caché si son anotadas con $@$Cacheable. Sin embargo, se 
puede utilizar lo siguiente:
    \begin{itemize}
    \item ALL. Forzar el caché para todas las entidades.
    \item NONE. Deshabilitado para todas las entidades.
    \item ENABLE\_SELECTIVE (default). Habilitar caché explícitamente marcado.
    \item DISABLE\_SELECTIVE. Habilitar caché a menos que se encuentre 
    explícitamente marcado con $@$Cacheable(false)
    \end{itemize}
\item \textbf{validation-mode}. Por defecto la validación Bean está activada. 
Cuando una entidad es creada, actualizada o borrada, es validada antes de 
mandarla a la base de datos. El esquema de la base de datos que es generado por 
Hibernate también refleja estas restricciones declaradas en la entidad. Se 
pueden hacer pequeños cambios si son necesarios, tales como:
\begin{itemize}
 \item AUTO. Si la validación bean está presente en el classpath, CALLBACK y 
DDL están activados.
 \item CALLBACK. Las entidades son validadas en su creación, actualización y 
borrado. Si no hay proveedor de validación, se lanzara una excepción a la hora 
de la inicialización.
 \item DDL. (No es estándar). El esquema de la base de datos son entidades que 
se validan en la creación, actualización y borrado. Si no hay proveedor de 
validación, se genera una excepción en el momento de la inicialización.
 \item NONE. No se utiliza validación.
\end{itemize}

\end{itemize}

\section{Conclusiones}
Es evidente la gran cantidad de parámetros que se pueden configurar en el 
archivo persistene.xml, sin embargo, al conocernos evita que uno se pueda 
sentir abrumado por todas las posibles configuraciones que se pueden lograr con 
tan solo modificar este archivo. Y al igual que con el archivo web.xml, el 
conocer el como funciona cada parámetro nos puede ahorrar tiempo y facilitar el 
proceso de desarrollo.

\end{document}
