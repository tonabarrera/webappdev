\documentclass[a4paper,12pt]{article}
\usepackage[utf8]{inputenc}
\usepackage[spanish]{babel}
\usepackage{graphicx}
\usepackage{url}
\usepackage{float}
\usepackage{listings}
\usepackage{xcolor}

%opening
\title{Tarea No. 7. Data Transfer Object}
\author{Barrera Pérez Carlos Tonatihu \\ Profesor: José Asunción Enríquez 
Zárate 
\\ Web Application Development \\ Grupo: 3CM9 }

\begin{document}

\maketitle
\newpage
\tableofcontents
\newpage

\section{Introducción}
El patrón DTO tiene como finalidad de crear un objeto plano (POJO) con una 
serie de atributos que puedan ser enviados o recuperados del servidor en una 
sola invocación, de tal forma que un DTO puede contener información de 
múltiples fuentes o tablas y concentrarlas en una única clase simple.

\section{Desarrollo}
Si bien un DTO es simplemente un objeto plano, sí que tiene que cumplir 
algunas reglas para poder considerar que hemos creado un DTO correctamente 
implementado \cite{oracle}:

\begin{itemize}
 \item Solo lectura: Dado que el objetivo de un DTO es utilizarlo como un 
objeto de transferencia entre el cliente y el servidor, es importante evitar 
tener operaciones de negocio o métodos que realicen cálculos sobre los datos, 
es por ello que solo deberemos de tener los métodos GET y SET de los 
respectivos atributos del DTO.
  \item Serializable: Es claro que, si los objetos tendrán que viajar por la 
red, deberán de poder ser serializables, pero no hablamos solamente de la 
clase en sí, sino que también todos los atributos que contenga el DTO deberán 
ser fácilmente serializables. Un error clásico en Java es, por ejemplo, crear 
atributos de tipo Date o Calendar para transmitir la fecha u hora, ya que 
estos no tienen una forma estándar para serializarse por ejemplo en 
Webservices o REST.
\end{itemize}

\subsection{Diferencias con Value object}
El patrón DTO fue llamado por error Value Object en la primera versión de Core 
J2EE Patterns. El nombre fue corregido en la segunda edición de este libro pero 
el error ya estaba hecho. Por lo que es importante aclarar las diferencias 
entre un DTO y un Value Object \cite{adam}.

\begin{itemize}
   \item Un DTO es un contenedor el cual es usado para trasportar datos entre 
capas y niveles. Principalmente contiene atributos. Se puede usar atributos 
publicos sin getters y setters. No contienen lógica del negocio y son 
serializables, lo cual es solo necesario si se transfieren DTOs a través de 
JVMs.
  \item Un Value Object representa un conjunto fijo de datos y es similar a un 
Java enum. Un Value Object no tiene ninguna identidad y es completamente 
identificado por su valor y es inmutable. Un ejemplo de esto seria Color.RED, 
Color.BLUE, SEX.FEMALE entre otros.
\end{itemize}

Sin embargo se siguen utilizando como sinónimos aunque esto este estrictamente 
mal.

\section{Conclusiones}
Los DTO son un patrón muy efectivo para transmitir información entre las 
diferentes capas de la aplicación, principalmente entre un cliente y un 
servidor, pues permite crear estructuras de datos independientes de nuestro 
modelo de datos. Además, nos permite controlar el formato, nombre y tipos de 
datos con los que transmitimos los datos para ajustarnos a un determinado 
requerimiento. Finalmente, si por alguna razón, el modelo de datos cambio (y 
con ello las entidades) el cliente no se afectará, pues seguirá recibiendo el 
mismo DTO.

\bibliographystyle{ieeetr}
\bibliography{referencias}

\end{document}
