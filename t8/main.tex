\documentclass[a4paper,12pt]{article}
\usepackage[utf8]{inputenc}
\usepackage[spanish]{babel}
\usepackage{graphicx}
\usepackage{url}
\usepackage{float}
\usepackage{listings}
\usepackage{courier}
\usepackage[T1]{fontenc}
\usepackage{xcolor}
\definecolor{gris}{RGB}{123, 126, 132}
\definecolor{morado}{RGB}{81, 40, 155}
\definecolor{amarillo}{RGB}{253,151,31}
\definecolor{magenta}{RGB}{249,38,114}

\lstdefinestyle{customJava}{
    frame=tb,
    language=Java,
    backgroundcolor=\color{white},   
    commentstyle=\itshape\color{gris},
    keywordstyle=\bfseries\color{magenta},
    numberstyle=\color{morado},
    stringstyle=\color{amarillo},
    identifierstyle=\color{black},
    basicstyle=\footnotesize,
    breakatwhitespace=false,         
    breaklines=true,                 
    captionpos=b,
    keepspaces=true,                 
    numbers=left,                    
    numbersep=5pt,                  
    showspaces=false,                
    showstringspaces=false,
    showtabs=false,                  
    tabsize=2,
}

\renewcommand{\lstlistingname}{Código}

%opening
\title{Tarea No. 8. Metodos}
\author{Barrera Pérez Carlos Tonatihu \\ Profesor: José Asunción Enríquez 
Zárate \\ Web Application Development \\ Grupo: 3CM9 }

\begin{document}

\maketitle
\newpage
\tableofcontents
\newpage


\section{Introducción}
En esta tarea se trabajaron tres métodos diferentes los cuales en un inicio 
pareciera que hacen lo mismo, sin embargo esto no tiene porque ser del todo 
cierto, es por esto que se explicara el como funciona cada uno de estos métodos 
y el funcionamiento que realizan.
\section{Desarrollo}

\begin{lstlisting}[language=Java, style=customJava, 
caption={Método Uno}, captionpos=b, basicstyle=\fontfamily{cmss}\small]
// El metodo recibe los parametros de request y response
public void metodoUno(HttpServletRequest request, HttpServletResponse response) 
{
        // Se trata de obtener una sesion asociada con el request
        // Si dicha sesion no existe se crea una ya que el parametro que se pasa
        // al metodo getSession es true
        HttpSession s = request.getSession(true);
        // Ya que tenemos la sesion introducimos un atributo de nombre "nombre" 
        // y con el valor "valor"
        s.setAttribute("nombre", "valor");
    }
\end{lstlisting}

\begin{lstlisting}[language=Java, style=customJava, 
caption={Método Dos}, captionpos=b, basicstyle=\fontfamily{cmss}\small]
// El metodo recibe los parametros de request y response
public void metodoDos(HttpServletRequest request, HttpServletResponse response) 
{
        /* Se trata de obtener una sesion asociada con el request
        * Si dicha sesion no existe se crea una ya que el parametro que se pasa
        * al metodo getSession es true
        */
        HttpSession s = request.getSession(true);
        /* Ya que la sesion es valida se intenta remover el atributo "nombre"
        * si no existe este atributo no pasa nada
        */
        s.removeAttribute("nombre");
        /*
        * Verificamos si la sesion no es nula, 
        * lo cual siempre es verdadero
        * debido a lo explicado con anterioridad
        */
        if (s != null){
        /* Ahora se invalida la sesion por lo cual 
        *  ya no se podra utilizar
        *  y se tendra que crear una nueva si es que se necesita
        */
            s.invalidate();
        }
    }
\end{lstlisting}

\begin{lstlisting}[language=Java, style=customJava, 
caption={Método Tres}, captionpos=b, basicstyle=\fontfamily{cmss}\small]
/* El metodo recibe los parametros de request y response
* el metodo tiene por tipo de retorno void, sin embargo en el metodo
* se regresan datos de tipo boolean, lo cual es un error
*/
public void metodoTres(HttpServletRequest request, HttpServletResponse 
response) {
       /* Se trata de obtener una sesion asociada con el request
        * Si dicha sesion no existe no se crea y retorna null 
        * ya que el parametro que se pasa
        * al metodo getSession es false
        */
        HttpSession s = request.getSession(false);
        /*Si la sesion es nula regresamos false*/
        if (s == null)
            return false;
        else {
        /* Se trata de obtener el atributo "nombre", en el caso
        * de que no exista se obtiene un null y por ende el metodo
        * regresa false en caso contrario regresa true
        * si su valor de "nombre" no es nulo
        */
            return s.getAttribute("nombre") != null;
        }
    }
\end{lstlisting}

\section{Conclusiones}
El analizar distintos tipos de situaciones en el desarrollo de aplicaciones es 
importe, esta tarea es una muestra de ello. El realizar este ejercicio de 
análisis permite reforzar el aprendizaje que se tiene.

\end{document}
